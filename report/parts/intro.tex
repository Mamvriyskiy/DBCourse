\chapter*{Введение}

В настоящее время технологии интернета вещей (IoT) становятся 
неотъемлемой частью нашей повседневной жизни, и одним из наиболее 
заметных примеров их применения является IoT-платформа для умного дома. 
Эта инновационная система объединяет различные устройства в доме, от 
светильников до умных термостатов, в единую сеть, которая может 
управляться и контролироваться через интернет.

Основной целью IoT-платформы умного дома является создание 
интеллектуальной инфраструктуры, способной адаптироваться к 
потребностям и предпочтениям пользователей. Она обеспечивает 
возможность контролировать освещение, отопление, кондиционирование 
воздуха, безопасность и другие аспекты жизни в доме с помощью смартфона 
или другого устройства с доступом в интернет.

Благодаря IoT-платформе умного дома пользователи могут не только 
управлять устройствами в реальном времени, но и получать данные и 
аналитику о потреблении ресурсов, обеспечивая оптимальное использование 
энергии и повышение эффективности.

С учетом быстрого развития технологий IoT и роста спроса на умные 
системы, платформы умного дома становятся все более интегрированными, 
расширяя свои возможности и предлагая новые функции для улучшения 
качества жизни пользователей.

Целью курсовой работы является разработка базы данных для IoT-платформы умный дом. 
Для достижения поставленной цели необходимо выполнить следующие задачи:

