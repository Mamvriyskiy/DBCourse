\chapter{Технологическая часть}

В данном разделе происходит выбор средств 
реализации базы данных и приложения, 
листинги кода, а так же будет показан интерфейс программы.


\section{Выбор системы управления базой данных}

В аналитическом разделе была выбрана реляционная 
система управления, наиболее распространнеными представителями
данной системы являются: 
Microsoft SQL, PostgreSQL, Oracle, а также MySQL. 

Выберем следующие критерии для сравнения выбранных систем:
\begin{enumerate}
    \item[1)] возможнсоть бесплатного использования;
    \item[2)] опыт работы с данной системой;
    \item[3)] наличие подробной документации. 
\end{enumerate}

Сравнение указанных систем управления базой данных(СУБД)
представлены в таблице \ref{tab:diff}:

\begin{table}[!ht]
    \centering
    \caption{\label{tab:diff} Сравнение СУБД по указаным критериям}
    \begin{tabular}{|l|l|l|l|l|}
    \hline
        СУБД & Microsoft SQL & PostgreSQL & Oracle & MySQL \\ \hline
        1 & - & + & - & +  \\ \hline
        2 & - & + & - & -  \\ \hline
        3 & + & + & + & +  \\ \hline
    \end{tabular}
\end{table}

По результатам сравнения для реляционной была выбрана PostgreSQL, так как это 
единственная система, которая удовлетворяет всем критериям.

\section{Выбор средств реализации}

Реализация части приложения,
которая обеспечивает доступ к базе данных, 
была выполнена с использованием языка программирования GoLang. 
Данный выбор обусловлен следующими причинами:

\begin{itemize}
    \item сборщик мусора, который позволяет автоматически
    освобождать память;
    \item статическая типизация, которая помогает 
    обнаруживать ошибки на этапе компиляции;
    \item встроенная поддержка параллельных вычислений, что
    позволяет делать разработку многопоточной;
\end{itemize}

Графический интерфейс был разработан с использованием HTML и CSS,
так же был выбран язык JavaScript для 
обработки событий и динамического обновления элементов.

Причина выбора данных средств заключается в следующем:
\begin{itemize}
    \item HTML и CSS поддерживаются всеми основными веб-браузерами, 
    что гарантирует доступность содержимого на различных устройствах;
    \item опыт работы с HTML и CSS;
    \item JS имеет возможность получать доступ к структуре HTML-документа, 
    что позволяет манипулировать элементами, стилями, атрибутами и событиями.
\end{itemize}

\section{Выбор среды разрабоки}

В качестве среды разработки был выбран Visual Studio Code. Данный выбор обусловлен
следующими причинами:
\begin{itemize}
    \item бесплатный доступ;
    \item поддержка выбранных языков;
    \item опыт работы в данной среде.
\end{itemize}

\section{Архитектура приложения}

Для разрабатываемого приложения была выбрана чистая архитектура, то есть приложение будет 
состоять из трех слоев: графический интерфейс, бизнес логика, доступ к данным.

Данная разделение имеет следующие преимущества:
\begin{itemize}
    \item приложение не зависимосит от используемых библиотек и фреймворков;
    \item удобство тестирования;
    \item возможность использования различных баз данных, так как
    бизнес логика не зависит напрямую от используемой базы данных;
\end{itemize}
% TODO: ссылка на чистую

\section{Реализация функции}

% TODO: описать работу функции


\section{Тестирование}



\section{Графический интерфейс}

