\chapter*{Введение}

В настоящее время технологии интернета вещей (IoT) становятся 
неотъемлемой частью нашей повседневной жизни, и одним из наиболее 
заметных примеров их применения является IoT-платформа для умного дома. 
Она объединяет различные устройства в доме, от 
светильников до умных термостатов, в единую сеть, которая может 
управляться и контролироваться через интернет.

Основной целью IoT-платформы умного дома является создание 
интеллектуальной инфраструктуры, способной адаптироваться к 
потребностям и предпочтениям пользователей. Она обеспечивает 
возможность контролировать освещение, отопление, кондиционирование 
воздуха, безопасность и другие аспекты жизни в доме с помощью смартфона 
или другого устройства с доступом в интернет.

С учетом быстрого развития технологий IoT и роста спроса на умные 
системы, платформы умного дома становятся все более интегрированными, 
расширяя свои возможности и предлагая новые функции для улучшения 
качества жизни пользователей.

Целью курсовой работы является разработка базы данных для IoT-платформы умный дом. 
Для достижения поставленной цели необходимо выполнить следующие задачи:
\begin{enumerate}
    \item провести анализ существующих решений;
    \item сформулировать требования к разрабатываемой базе данных;
    \item проанализировать существующие базы данных на основе данной задачи;
    \item спроектировать и разработать базу данных;
    \item спроектировать и разработать приложение для взаимодействия с базой данных;
    \item провести исследование зависимости времени выполнения запросов
    с использованием индексов и без.
\end{enumerate}
